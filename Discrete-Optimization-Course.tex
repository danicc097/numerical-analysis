\documentclass[]{article}

\usepackage{amsmath}
\usepackage[margin=1in]{geometry}

\title{Discrete optimization}
\author{}


\begin{document}

\tableofcontents

\maketitle
\section{Knapsack}
\subsection{Greedy algorithms}
Many possible implementations to solve the same problem. For knapsack in
particular we could come up with: more valuable is better, more items is better, more value per kg is
better.

Zero guarantees that the problem is optimally solved. The problem needs to be
easy. We can use a quick and dirty greedy algorithm to get started with a
problem.

\subsection{Modelling}

How to model, in general:
\begin{enumerate}
  \item Select some \textbf{decision variables}, which encode the result we are interested
        in.
  \item Express the \textbf{constraints} in terms of the decision variables.
  \item Declare an \textbf{objective function}, which specifies the quality of each solution.
\end{enumerate}

As a result, the optimization model we get is declarative (\textit{what}, not \textit{how}), therefore there may be multiple ways to solve it.

Applying this to the knapsack problem:

\begin{enumerate}
  \item{Decision variables}:
  \begin{itemize}
    \item $x_i$ denotes whether item $i$ is selected.
    \item $x_i=1$ means its selected.
    \item $x_i=0$ means its not selected.
  \end{itemize}
  \item{Constraints}:
  \begin{itemize}
    \item selected items cannot exceed the capacity for a set of items $S$:
          $$
            \sum_{i \in S}w_i x_i \leq K
          $$
  \end{itemize}
  \item{Objective function}: total value of the selected items:
  $$
    \sum_{i \in S}v_i x_i
  $$
\end{enumerate}

This leaves us with the following model:

\centering
\def\arraystretch{4} %  1 is the default, change whatever you need
\begin{tabular}{ll}
  minimize   & $
    \displaystyle\sum_{i \in S}v_i x_i
  $              \\
  subject to & $
    \begin{align}
      \sum_{i \in S}w_i x_i \leq K
      \\
      x_i \in \{0,1\} \ (i \in S)
    \end{align}
  $              \\
\end{tabular}

\end{document}
